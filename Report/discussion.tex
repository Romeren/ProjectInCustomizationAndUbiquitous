\label{discussion}
DashThinks has become a language which let you easily setup a Web-page with graphs and a network
of pages linking back and forth to each other.
In DashThinks it is possible to define where the data is coming from and select data from that
endpoint. 
Combining different data is possible using standard mathematical formulas%, but currently only if the different data has the same amount of entries
.

Given more time to work on the language additional features would be added to the language.
These features would make the language even more convenient and add more value to the language.

\subsection{Inbound Data stream} 
The concept of having external devices post data to the server and save that data with a
date stamp on the server, would allow DashThinks to work as a datacenter where sensors can
post and the users could easily see the data coming in using the pages defined. 
Further more, would it also allow apps on smart phones as sources for data collection.

\subsection{Event Handlers} 
Allowing definitions of Events on the data sources could make it possible to build context
aware systems from DashThinks. 
These Event Handlers could be able to contain executable code that would allow the
server to warn or inform users of any irregularities or just give out common reports to
a mailing list.
In addition, it could also be triggering an actuator and controlling a buildings environment.

\subsection{Live Updated Graphs}
Current implementation contains a static graph which is only updated once the site is
refreshed, and at that point it calls the external data sources for data at every refresh,
which is not efficient. Live updated graphs would require that the server kept a cache available
for each EndPoint and that it would dynamiccally update the graphs on the webpage. 
Additional work with web sockets and refactoring the webpage template would be required. 

\subsection{Comparing Formula and GNUPlot} 
\label{gnuplot}
Where GNUPlot\footnote{see \cite{GNUPlot}} is a language used to define
plots, having its own DSL to define these graphs. GNUPlot has some similarities to how DashThings's
DSL is defined, but it also contains a lot of further features, some of these are configurations
for defining the color of a line, which in DashThings are chosen by random. It contains a way
to define and use formulas like possible in DashThings, but it has the basic functions like
Sin, Cos and the like. GNUPlot was first discovered once the actual development of the language
was done and therefore no inspiration from it was added to the language, which could possible have
made the language more customizable and smarter than it's current state.
\\
\\
Some of the functionalities which could be inherited from GNUPlot would be the possibilities to
define color of lines, define legend names for each line for easy distinguishing which line
correspond to what type of value and where that value comes from. Additional options for GNUPlot
include their sophisticated formula system for defining formulas for use in plotting an array,
this can somewhat be archived in DashThings as well, but is concealed by the data sources, where
a data source actually works as a function. To give an example as seen in Listing
\ref{lst:datasource-formula} but as spotted only very simple functions can be applied using
this method. A means to make this better would be to include more options within the formula
language, to implement some of the methods also available in GNUPlot like sin and cos, but also
include functions which just evaluate to a single number that is applied to the entire array
like SUM and AVERAGE.

\subsection{Easy Multi data handling} 
Currently there is some issues with how highly dimensional data is handled, where a lot of
duplication code is made.
Examples of this problem have been provided in Appendix \ref{laguagedificulties}.
For instance, the systems handles making advanced queries to external sources which may response
with multiple data streams in an array. 
But to specify how to parse the response to the system, it is necessary to specify a data
selector for each entry in the array, which generates a lot of duplicated code and is annoying
for the user using the language.
However a solution for this problem have already been thought of, but due to time limitation
it have not yet been implemented. Instead it will now be described.
The problem starts at the selectors in the SchemaParser, but by adding a key word, for instance
"asArray", to the language when specifying a step would be enough to solve the problem.
For the dimension, the problem may be solved by allowing users to specify a for-each loop, and
making it possible to iterate over dimensions.
Lastly in the individual function specification can be solved by implementing some of the
functionality found in subsection \ref{gnuplot}, Comparing Formula and GNU-Plot. 

\subsection{Automatic Data source Display on Web-pages} 
Currently it is required to define
which data sources is displayed on a page, this approach is arguable suitable as you always
know what each page contains. But it might be more interesting to show content in a more
intelligent way. For the OU44 data it might be interesting to make a functionality which could
map graphs to pages based on keywords like the room number. Also using this method define a
template page in the DSL and when give it a list of id's which would be mapped to separate pages
containing the same types of data defined in the template, but given the id would have different
data shown on each page.

\begin{figure}
  \caption{Datasource showing usage of a formula}
  \label{lst:datasource-formula}
  \lstinputlisting[language=Java, frame=single, breaklines=true,tabsize=2]{code/datasource-formula.vis}
\end{figure}

\subsection{Internal vs External DSL} The current solution chosen for the DSL DashThings has
been an External DSL using XText and Xtend. But a lot of the code which is generated by the DSL
could easily have been archieved by using an internal python DSL instead, and that might in a lot
of the cases be easier than implementing the current external DSL.
 