HERE GOES DISCUSSION !!! !

DashThinks has become a language which let you easily setup a Webpage with graphs and a network
of pages linking back and forth to each other.
In DashThinks it is possible to define where the data is coming from and select data from that
endpoint. Combining different data is possible using standard mathematical formulas,
but only if currently only if the different datas has the same amount of entries.

Given more time to work on the language additional features would be added to the language.
These features would make the language even more convenient and add more value to the language.

\subsection{Inbound Datastream} is the concept of having external devices post data to the
server and save that data either with a date stamp from the server run from the DashThinks
program or with a date stamp already given by the posting device. This would allow the a program
from DashThinks to work as a datacenter where sensors could post and the users could easily see
the data coming in using the pages defined.

\subsection{Event Handlers} which could be defined on the datasources defined in a DashThinks
program. These Event Handlers would be able to contain executeable code that would allow the
server to warn or inform users of any irregularities or just give out common reports to
a mailing list. Not that a mailing list would be the only option, it could also be that it
should trigger an external url which could then do additional work with whatever that external
server would do.

\subsection{Live Updated Graphs} on the webpages. Current implementation contains a static
graph which is only updated once the site is refreshed, and at that point it calls the
external data sources for data at every refresh, which is not super efficient. This feature
would require that the server kept a cache available for each EndPoint and that dynamic
update of the graphs would happen on the webpage. Additional work with web sockets and
rewritting of the webpage template would be required.

\subsection{Easy Multidata handling} Currently there is some issues with how highly dimensional data is handled, where a lot of reperative typing has to be done.
For this some kind of looping or a ``catch all iterater'' would increase the ease of use and
reduction to the amount of code required for a program to run. The issue at hand is seen in the
definition of dimensions on datasources and for selectors on GetEndPoints, where for some examples
12 data arrays from one url are handled in the exact same way. Instead a prefix or foreach loop
could be declared for these scenarios to chop it into 1 line for those specific cases, rather
than using 1 line for each data array in source.

\subsection{Automatic Datasource Display on Webpages} Currently it is required to define
which data sources is displayed on a page, this approach is argueable suitable as you always
know what each page contains. But it might be more interesting to show content in a more
intelligent way. For the OU44 data it might be interesting to make a functionality which could
map graphs to pages based on keywords like the room number. Also using this method define a
template page in the DSL and when give it a list of id's which would be mapped to seperate pages
containing the same types of data defined in the template, but given the id would have different
data shown on each page.

\subsection{Metadata multicall}

\subsection{Comparing Formula and GNUPlot} where GNUPlot is a language used to define
plot having its own DSL to define these graphs. GNUPlot has some simularities to how DashThings's
DSL is defined, but it also contains a lot of further features, some of these are configurations
for defining the color of a line, which in DashThings are chosen by random. It contains a way
to define and use formulas like possible in DashThings, but it has the basic functions like
Sin, Cos and the like. GNUPlot was first discovered once the actual development of the language
was done and therefore no inspiration from it was added to the language, which could possible have
made the language more customizeable and smarter than it's current state.

\subsection{Data Cleaning}