The first step towards automatization of buildings and homes with Internet of Things is to understand the data provided. 
Through this understanding it is then possible to decide actions to take based on one or more sensor input given. 
Making decisions and taking action for one building is not generalizable enough to apply to other buildings without customization. 
Further are actions rooted in a problem that needs to be solved. 
This means that customization is needed whenever one of the following is true:
\begin{itemize}
\item Problems that needs to be solved might differ from one building to  another building
\item The buildings infrastructure might not be the same between buildings meaning that the solution might differ
\item Context of the environment have changed
\end{itemize}

In general there is a great need for customization or customizable application in building  automatization through Internet of Things.

Using charts or graphs to visualize large amounts of complex data is easier than poring over the data itself. 
Data visualization is a quick, easy way to convey concepts in a universal manner. Data visualization can be used for different purposes. 
Some of these are:
\begin{itemize}
\item Decision making
\item Identify areas that need attention or improvement
\item Clarify which factors influence behaviuor
\item Identify behaviour or patterns
\end{itemize}

The problem is that a visualisation is not universal in sense of the purposes it can be used for. 
Therefore, there are a great need for customized or customizable visualizations that supports the different purposes. 

The purpose for the project is to have a website, with responsive design, which can be viewed on any browser like Mobile, Desktop, Laptop and tablet. 
The pages on the website will contain Links between the pages on the webpage. 
It will also be possible to link to external webpages. 
Additionally the pages will contain graphs and tables which shows data. 
The data will either be fetched from external sources or external sources will have the possibility to post data to a data source given a predefined data schema. 

Data in data sources can be transformed using formula expressions and have to go through at least 1 formula expression in order to be displayed on a graph. 
This expression will be on the graph. 

In order to enable the user to create the above in a easy customizable way three self made language will be used.