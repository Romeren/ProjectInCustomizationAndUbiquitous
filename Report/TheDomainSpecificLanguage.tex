
\subsection{Overview}
The DSL is divided into three components, a arithmetic expressions language, a language for specifying a web site and a language for specifying input sources for the system. 
The expression language is a utility language which is included by the to others. 
This language defines how a user can create arithmetic expressions in either of the other two languages. 
The Web visualizer is a language to create dynamic web sites, like the sample page shown on the front page. 
This language lets the user create pages on a site, link different pages to each other and visualize information through the use of graphs. 
The Datasources language, which is still to be developed, will be used to specify different data sources, like for instance a external databases or APIs. 
In addition this language will also be used to specify internal persistence and a API for other systems to use for posting data to the Dashk system.

\subsection{The Formula Expression Language}
The Formula Expressions DSL is responsible for understanding mathematical formulas, which will be applied to tables of data.
\paragraph{Meta model}
\paragraph{Language Validation}
\paragraph{Code Generation}


\subsection{The Web visualizer Language}
For generating dashboards custom language have been created.
This language specify:
\begin{itemize}
\item Which pages the web-interface consist of
\item The navigation between pages
\item Which data to display and where
\item Where and how to obtain the data
\subitem Internal data streams
\subitem External data streams
\item How to manipulate the data streams
\end{itemize}
The following subsection will look into detail of how the language is structured and the syntax and grammar of the language.

\subparagraph{The Grammar}

Figure \ref{fig:languagemodel} shows the structure of the language.
This model shows that the language defines of a system containing three parts:
\begin{itemize}
\item Pages
\item Sources
\item Schemas
\end{itemize}
\textbf{The Pages,} is what the language in turns will use to create web pages in a system defined in the language.
A page is a collection of tiles. 
Each tile is an extendible definition of a visual object on a page.
In the current version, there are two types of Tiles, Links and Graphs.
A Link definition contains a pointer to a page. 
A Graph however, has a reference data \textit{source} from witch it needs to fetch data from.
\begin{figure}
\begin{center}
\includegraphics[width=\linewidth]{images/GrammaDefinition}
\end{center}
\caption{Sample grammar definition}
\label{fig:grammadefinition}
\end{figure}
Figure \ref{fig:grammadefinition} shows how this grammar was specified.
\textbf{The Sources, } can be of two different types, an \textit{EndPoint} or a \textit{DataSource}.
An EndPoint is intended to function as interface towards external systems or data sources. 
Whereas a Datasource is an internal definition intended for filtering, data manipulation or data grouping.\\
This Source, Endpoint, Datasource structure have been defined as a compositional pattern, with source as the component, the Endpoint as a leaf and the Datasource as the composite.
This pattern makes the it possible to compose any number of different structures.
\textbf{Schemas}
\begin{figure}
\begin{center}
\includegraphics[width=\linewidth]{images/languagemodel}
\end{center}
\caption{Model of the language}
\label{fig:languagemodel}
\end{figure}



Below table shows the grammar for the web visualizer language to the left and an example of use on the right. 
As seen from the left side the system consist of a number of pages which again consists of a number of tiles. 
One tile can, in current implementation be either a link to another page or a definition of a graph. 
Since the datasource language have not yet been implemented the web visualizer language currently work with a hard-coded testset. 
Further are only one graph type implemented (a line chart), however the system makes use of D3 framework meaning that extending the system with other graph types is a smaller task.

\begin{figure}
\begin{center}
\includegraphics[width=\linewidth]{images/ExampleUsePages}
\end{center}
\caption{Example language use, defining pages}
\label{fig:examplePages}
\end{figure}

\begin{figure}
\begin{center}
\includegraphics[width=\linewidth]{images/ExampleUseDataSources}
\end{center}
\caption{Example language use, defining data sources}
\label{fig:exampleDatasources}
\end{figure}

\begin{figure}
\begin{center}
\includegraphics[width=\linewidth]{images/ExampleUseEndPoint}
\end{center}
\caption{Example language use, defining endpoints}
\label{fig:exampleEndpoints}
\end{figure}

The example on the right site creates a System instance with three pages, each page with a number of Links and Graphs.

\paragraph{Meta model}
Below is the first meta model of the system. 
This model matches the language specification one to one. 
The System once again is a collection of pages, which again consists of a number of tiles which can be of the two previous described types. 

\paragraph{Language Validation}

\paragraph{Code generation}
To create the website the project uses the Django framework. 
This framework uses an MVC architektur which is extended with a template pattern with HTML documents. 
In order to make the output easy readable and customizable for a user, the first meta model needs to be mapped to a second meta model with more explicit information which matches the MVC architecture better.
This second model can be seen in below diagram. 

\begin{figure}
\begin{center}
\includegraphics[width=\linewidth]{images/websitemodel}
\end{center}
\caption{Model of the Website}
\label{fig:websitemodel}
\end{figure}

The first step of mapping the previous model into the second meta model, is to create a system with a configuration. 
Secondly a default controller is added with the default Django Admin configuration pages. 
This controller gets a number of URLs and default views. 
After this a second controller gets added, this controller is used for the pages defined in the language. 
Each page gets transformed into a url entity, view entity, model and a template HTML-file.
The HTML file holds a easy customizable structure of the content added to the page. The view and model will in time be connected to the Data sources from the previous specified language. 
In addition will the view of a page be responsible of rendering the template with context and data from a datasource. 
After this model have been created each entry in the model is looped through and generated into a file.


\begin{figure}
\begin{center}
\includegraphics[width=\linewidth]{images/PackageDiagram}
\end{center}
\caption{Package diagram of generated sources}
\label{fig:packagediagram}
\end{figure}
\subparagraph{Pages}
\subparagraph{Data sources}
\subparagraph{Api}
